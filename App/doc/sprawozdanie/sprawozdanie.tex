\documentclass{article}
\usepackage[utf8]{inputenc}

\title{Sprawozdanie Programowanie Komponentowe}
\author{Kacper Ochnik}
\date{\today}

\begin{document}

\maketitle

\section*{1. Opis Zadania Programistycznego}

Niniejsze sprawozdanie opisuje implementację aplikacji komponentowej która składa się z komponentu kalkulatora oraz aplikacji testującej.
Natomiast kalkulator posiada komponent z funkcjami matematycznymi, takimi jak dodawanie, odejmowanie, mnożenie i dzielenie.

\section*{2. Aplikacja - Kalkulator}

Aplikacja skupia się głównie na funkcjonalności kalkulatora, implementowanej w komponencie kalkulatora.

\section*{3. Komponent Kalkulator – Ogólny Opis}

Komponent kalkulatora został zaprojektowany do obsługi podstawowych operacji matematycznych i przechowywania wyników.

\section*{4. Zmienne Prywatne i Ich Znaczenie}

Zmienne prywatne komponentu to m.in. liczba\_1, liczba\_2, operacja, wynik, które przechowują odpowiednie wartości.

\section*{5. Funkcje Dostępowe do Zmiennych Prywatnych}

Komponent udostępnia funkcje dostępowe, takie jak getLiczba1(), getLiczba2(), getWynik(), które umożliwiają dostęp do prywatnych zmiennych.

\section*{6. Przykłady Wywołań Setterów/Getterów Komponentu}

Przykładowe wywołania getterów i setterów można znaleźć poniżej:

\begin{verbatim}
kalkulator.setLiczba1(10);
kalkulator.setLiczba2(5);
int wynik = kalkulator.getWynik();
\end{verbatim}

\section*{7. Serializacja}

Komponent obsługuje serializację danych, co umożliwia zapis i odczyt stanu kalkulatora.

\section*{8. Cechy Komponentu plus Fragmenty Kodu}

Komponent posiada funkcje do dodawania, odejmowania, mnożenia i dzielenia. Poniżej fragment kodu operacji dodawania:

\begin{verbatim}
public int dodaj() {
    wynik = liczba_1 + liczba_2;
    return wynik;
}
\end{verbatim}

\section*{9. Testy Komponentu (Przypadki Testowe)}

Przeprowadzono testy jednostkowe, sprawdzające poprawność działania każdej z operacji kalkulatora.

\section*{10. Aplikacja Testująca Komponent}

Stworzono prostą aplikację testującą, która wykorzystuje komponent kalkulatora do przeprowadzania różnych operacji.

\section*{11. Przykład Wykorzystania Komponentu w Wybranej Aplikacji}

Komponent kalkulatora został użyty w aplikacji do obsługi obliczeń matematycznych, np. przy fakturach.

\section*{12. Wnioski}

Implementacja kalkulatora jako komponentu jest skutecznym rozwiązaniem umożliwiającym łatwe dodawanie funkcji matematycznych do aplikacji. Testy potwierdzają poprawność działania komponentu.

\end{document}
